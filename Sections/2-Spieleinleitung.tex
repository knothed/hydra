\section{Spieleinleitung} \label{Spieleinleitung}

\subsection{Verteilen der Karten}
\rules{
\regel Die 50 Karten werden vom Geber gleichmäßig an die fünf Spieler verteilt.
\regel Das Austeilen verläuft in drei Runden: zuerst werden jedem Spieler drei Karten ausgeteilt, danach vier, danach drei.
\regel Der Geber beginnt jede Runde beim Spieler nach dem Geber und teilt die Karten den Spielern im Uhrzeigersinn aus.
}

\subsection{Ansagen}
\rules{
\regel Nach dem vollständigen Austeilen sagen die Spieler, beginnend beim Spieler nach dem Geber, der Reihe nach an, ob sie ein Vorbehaltsspiel (\ref{Vorbehalte}) spielen möchten.
\regel Im Falle eines Vorbehaltes sagt ein Spieler \textit{Morbus}, andernfalls \textit{Valeo}.
\subsubsection*{Armut}
\regel Hat ein Spieler maximal einen Standardtrumpf (\ref{Standardtrümpfe}) oder mindestens fünf 9en auf der Hand, so kann er \textit{Armut} ansagen.
\regel Armut muss angesagt werden, bevor der betroffene Spieler \textit{Morbus} oder \textit{Valeo} ansagt.
\regel Im Falle einer angesagten Armut wird das Spiel wiederholt. Der Geber bleibt gleich, die Karten werden neu ausgegeben und das Ansagen beginnt von vorn.
}

\subsection{Spieltypen}
\rules{
\regel Es gibt drei Spieltypen: Soli, Anträge und gesunde Spiele.
\regel \label{Vorbehalte} Soli und Anträge zählen zu den Vorbehaltsspielen.
\subsubsection*{Spieltypfindung}
\regel Haben alle Spieler \textit{Valeo} angesagt, wird ein gesundes Spiel gespielt.
\regel Andernfalls spezifizieren alle Spieler, die \textit{Morbus} angesagt haben, der Reihe nach die Art ihres Vorbehaltes: \textit{Solo} oder \textit{Antrag}. Es beginnt der Spieler nach dem Geber.
\regel \textit{Antrag} kann nur angesagt werden, wenn der entsprechende Spieler zwei Pik-Damen oder zwei Kreuz-Damen auf der Hand hat.
\regel Haben alle Spieler \textit{Antrag} angesagt, so wird ein Antrag gespielt. Hat mindestens ein Spieler \textit{Solo} angesagt, so wird ein Solo gespielt.
\regel Der erste Spieler nach Ansagereihenfolge, welcher den zustandegekommenen Spieltyp (Solo oder Antrag) angesagt hat, erhält das Spiel:
\begin{itemize}
	\item Bei einem Solo wird dieser Spieler zum Solisten.
	\item Bei einem Antrag wird dieser Spieler zum Antragsteller und führt nun den Antrag aus.
\end{itemize}
\subsubsection*{Ausführung des Antrags}
\regel Der Antragsteller legt entweder beide Pik-Damen oder beide Kreuz-Damen verdeckt ab und legt eine dritte Karte offen hinzu.
\regel Nun sagen die Spieler reihum, beginnend beim Spieler nach dem Antragsteller, an, ob sie den Antrag annehmen, bis entweder ein Spieler den Antrag annimmt oder alle vier Mitspieler den Antrag ablehnen.
\regel \label{AngenommenerAntrag} Nimmt ein Spieler den Antrag an, so wird er zum Antragnehmer. Er nimmt die drei vor dem Antragsteller liegenden Karten auf und gibt ihm danach drei Karten zurück.
\regel Nimmt kein Spieler den Antrag an, so nimmt der Antragsteller die vor ihm liegenden Karten wieder auf und das Spiel wird zu einem gesunden Spiel.
}

\subsection{Parteistellung}
\rules{
\regel Abhängig vom Spieltyp und der Kartenverteilung gibt es entweder zwei oder drei Parteien, die gegen die jeweils anderen Parteien spielen.
\regel Beim Solo bildet der Solist eine Partei und die vier Gegenspieler bilden eine Partei.
\regel Bei einem angenommenen Antrag (\ref{AngenommenerAntrag}) bilden der Antragsteller und der Antragnehmer eine Partei, und die drei Gegenspieler bilden eine Partei.
\regel Bei einem gesunden Spiel hängt die Parteieinteilung von der Verteilung der vier Pik- und Kreuz-Damen ab:
\begin{itemize}
	\item Die ein bis zwei Spieler, welche die Pik-Damen besitzen, spielen in einer Partei.
	\item Die ein bis zwei Spieler, welche die Kreuz-Damen besitzen, spielen in einer Partei.
	\item Bei Überschneidungen zwischen so entstandenen Parteien werden diese zusammengelegt. (Beispiel: ein Spieler besitzt sowohl Kreuz- als auch Pik-Dame. Dieser bildet nun eine Partei mit dem oder den Spielern, welche Kreuz- oder Pik-Dame besitzen.)
	\item Alle Spieler, welche weder Pik- noch Kreuz-Dame besitzen, bilden eine Partei.
\end{itemize}
}