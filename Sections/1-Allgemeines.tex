\section{Allgemeines}
\rules{
\regel Hydra ist ein Kartenspiel für fünf Spieler.
\regel Hydra kann zu sechst gespielt werden, indem der Geber vom jeweils aktuellen Spiel ausgeschlossen wird.
}

\subsection{Spielkarten}
\rules{
\regel Das Hydrablatt besteht aus 50 Spielkarten.
\begin{itemize}
	\item Kartenwerte: \textit{9}, \textit{10}, \textit{Bube}, \textit{Dame}, \textit{König}, \textit{Ass}
	\item Farben: \textit{Karo}, \textit{Herz}, \textit{Pik}, \textit{Kreuz}
\end{itemize}
Jede Kombination aus Kartenwert und Farbe ist zweimal im Deck vorhanden. Zusätzlich befinden sich zwei Joker (genannt \textit{Hydra}) im Deck.
\regel Zwei Karten mit gleichem Kartenwert und gleicher Farbe, oder zwei Hydren, sind nicht voneinander unterscheidbar.
}

\subsection{Spielformat}
\rules{
\regel Ein \textit{Spiel} bezeichnet den Prozess der Spieleinleitung (Abschnitt \ref{Spieleinleitung}: Geben der Karten, Ansagen) und der Spieldurchführung (Abschnitt \ref{Spieldurchführung}: Ausspielen aller Karten).
\regel Werden mehrere Spiele durchgeführt, bewegt sich die Rolle des Gebers nach jedem vollendeten Spiel einen Platz im Uhrzeigersinn fort.
\regel Die Spielrichtung ist der Uhrzeigersinn. \textit{Nächster Spieler} und ähnliche Ausdrücke beziehen sich immer auf den nächsten Spieler im Uhrzeigersinn.
}